% !TeX root=main.tex
\pagenumbering{arabic}
\chapter{مقدمه}
\thispagestyle{empty}
	در سال‌های اخیر و با پیشرفت‌های چشمگیر در حوزه هوش مصنوعی، پردازش زبان طبیعی و پردازش تصویر مسئله‌هایی با کاربرد عملی در زندگی روزمره انسان‌ها طراحی شده است. یکی از مواردی که اخیرا مورد توجه قرار گرفته است، بحث پرسش و پاسخ تصویری می‌باشد. این مسئله کاربرد‌‌‌های زیادی در کمک به نابینایان، دستیار هوشمند و موارد مشابه می‌تواند داشته باشد.	  
%\section{اهمیت و کاربرد مسئله}
\newline
با توجه به اهمیت بحث پرسش و پاسخ تصویری در کمک به افراد کم‌بینا یا نابینا در زندگی روزمره و کمک به بهبود و تسهیل امور جاری روزانه، استفاده از مدل‌های آموزش دیده بر روی تلفن همراه یا وب‌سایت‌ها در قالب نرم‌افزار‌های کاربردی از اهمیت بالایی برخوردار است. از سوی دیگر اغلب تلفن‌های همراه قدرت پردازش و حافظه محدودی دارند. استفاده بهینه از منابع موجود بسیار حائز اهمیت است. بنابراین علاوه بر آموزش مدل مناسب که برای این مسئله به دقت قابل قبولی برسد، لازم است مدل ساخته شده از حجم مناسبی برخوردار بوده و قابل استفاده بر روی تلفن همراه با استفاده از کمترین منابع باشد. به طوری که کارکرد تلفن همراه را دچار اختلال نکند. 
\newline	 
با گسترش استفاده از شبکه‌های ترنسفورمر دقت‌های به دست‌آمده در مسئله پرسش و پاسخ تصویری به مقدار قابل قبولی رسیده است. اما شبکه‌های ترنسفورمری اغلب تعداد پارامتر‌های بالایی دارند.از این رو کوچک کردن مدل و فشرده سازی آن از جمله مسائل داغ مورد بررسی است.
در این پژوهش سعی شده است که هرس شبکه عصبی بر روی مسئله پرسش و پاسخ تصویری مورد بررسی قرار بگیرد و نتایج مدل فشرده شده با مدل اصلی مقایسه گردد. همچنین تاثیر فشرده سازی بر دقت مدل کاهش یافته و عملکرد آن بررسی ‌شود.

